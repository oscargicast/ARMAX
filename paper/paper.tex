\documentclass[twocolumn]{IEEEtran}

\usepackage[ansinew]{inputenc} 
\usepackage{amsmath}
\usepackage{graphicx}
\usepackage{graphics}
\usepackage{hyperref}
\usepackage{longtable}               
\usepackage{fancyhdr}
\usepackage{times}
\usepackage{color}                   
\usepackage{makeidx}                 
\usepackage{cite}
\usepackage{float}
\setcounter{page}{1}

\begin{document}

\title{Identificaci�n de una planta de 3 Opams (sistema de tercer orden) utilizando la estructura param�trica ARMAX.}


\author{Autores \\ 
				Estrada Vidal, Jorge  \textcolor{blue}{jor1550g@gmail.com} \\
				Florian Chacon, Erick  \textcolor{blue}{erick.florian.uni@gmail.com} \\
				Giraldo Castillo, Oscar \textcolor{blue}{oscar.gi.cast@gmail.com} \\ 		
				\vspace{4 mm}
				Asesores: \\
				Ing. Rodriguez Bustinza, Ricardo \textcolor{blue}{robust@uni.edu.pe} \\ 	
				
				\vspace{8 mm}
				\emph{Universidad Nacional de Ingenier\'ia}
		}			

		
		
%\markboth{IEEE Trans...}{Murray and Balemi: ...}
\maketitle





\section{OBJETIVOS} %%%%%%%%%%%%%%%%%%%%%%%%%%%%%%%%%%%%%%%%%%%%%%%%%%%%%%%%%%%%%%%%%%%%%%%%%%%%%%%%%%%%%%%%%%%%%%%%%%%%%%%%%

$\big(t\big) + a_{1} y \big(t-1\big)$

\PARstart{E}{l} Consejo Nacional para la Integraci�n de la Persona con Discapacidad (CONADIS \cite{uno}) es un organismo p�blico que conjuntamente con el INEI realiz� en el 2008 un censo nacional mostrando que del total de inscritos en el Registro Nacional por tipo de limitaci�n, la mayor parte de de ellas es de locomoci�n con 26 106, seguida por destreza con 25 765 del total de personas registradas en el Per� (ver Fig. ~\ref{tabla1}). Por lo cual, pensamos que la tecnificaci�n mediante m�todos de \textbf{bioingenier�a} de bajo costo aplicados en los campos de terapia f�sica  por medio de videojuegos son necesarios para detener el crecimiento de la poblaci�n con limitaciones motrices de bajos recursos. Por ello, haciendo uso de las gran gamma de sensores de bajo costo que hoy en d�a nos brinda un smartphone (tales como giroscopio, aceler�metro y giroscopio. Ver Fig. ~\ref{PhoneMotion}) daremos soluci�n a esta problematica.




\section{MARCO TE�RICO}%%%%%%%%%%%%%%%%%%%%%%%%%%%%%%%%%%%%%%%%%%%%%%%%%%%%%%%%%%%%%%%%%%%%%%%%%%%%%%%




\section{PRESENTACI�N DE RESULTADOSS} %%%%%%%%%%%%%%%%%%%%%%%%%%%%%%%%%%%%%%%%%%%%%%%%%%%%%%%%%%%%%%%%%%%%%%%%%%%%%%%%%%%%%%%%%%%%%%%%%%%%%%%%%%

Hoy en d�a, el costo de los smartphones es cada vez menor. Esto facilita la adquisici�n de uno y por lo tanto el alcance de nuestra interfaz, ya que esta es compatible con cualquier dispositivo con el sistema operativo Android. Adem�s, actualmente los smartwatch (basados en la misma tecnolog�a que los smartphone) se vienen desarrollando y prometen ser un boom principalmente por su potabilidad, sin embargo no los empleamos en nuestra aplicaci�n ya que aun no cuentan con todos los sensores necesarios para el reconocimiento de movimientos que deseamos realizar. A pesar de ello, esto no es un problema puesto que el software que realizaremos en esencia no cambia, y podr�a ser implementado en un smartwatch, o en hardware propio el cual desarrollaremos seg�n el segmento de mercado que queramos ocupar. 


\section{CONCLUSIONES} %%%%%%%%%%%%%%%%%%%%%%%%%%%%%%%%%%%%%%%%%%%%%%%%%%%%%%%%%%%%%%%%%%%%%%%%%%%%%%%%%%%%%%%%%%%%%%%%%%%%%%%%%%%%%%%%%%%%%%%%%

\begin{itemize}
	\item La proyecci�n de �ngulos a la pantalla es de forma nolineal por realizarse radialmente, es por ello que en una primera etada del proyecto se presentaban peque�os saltos en la proyecci�n del puntero, ya que habiamos considerado una transformaci�n del tipo lineal.
	
 	\item Comunicacion PC-Smartphone v�a TCP/IP. (ver Fig. ~\ref{tcp}) 
 	
\end{itemize}


\bibliographystyle{IEEE} %%%%%%%%%%%%%%%%%%%%%%%%%%%%%%%%%%%%%%%%%%%%%%%%%%%%%%%%%%%%%%%%%%%%%%%%%%%%%%%%%%%%%%%%%%%%%%%%%%%%%%%%%%%%%%%%%

\nocite{*}
\bibliographystyle{IEEE}

\begin{thebibliography}{1}

\bibitem{uno}
Ministerio de Salud, Per�
\newblock {\em REGISTRO NACIONAL DISCAPACIDAD EN CIFRAS}
\newblock CONADIS-INEI 2008

\bibitem{git}
bitbucket.org
\newblock {\em http://git-scm.com/}
\newblock 

\bibitem{bitbucket}
bitbucket.org
\newblock {\em https://bitbucket.org/jorgenro/proyecto-mecatronico}
\newblock Repositorio privado

\end{thebibliography}




\end{document}
